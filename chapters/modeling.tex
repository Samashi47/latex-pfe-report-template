\newpage
\chapter{Modeling}
\section{Introduction}
\lipsum[1]
\section{Use Case Diagram}
\lipsum[2-3]

\begin{figure}[H]
    \centering
    \begin{adjustbox}{width=\textwidth, keepaspectratio}
        \begin{tikzpicture}
            \begin{umlsystem}[x=4, fill=red!10]{The system}
                \umlusecase{use case1}
                \umlusecase[y=-2]{use case2}
                \umlusecase[y=-4]{use case3}
                \umlusecase[x=4, y=-2, width=1.5cm]{use case4 on 2 lines}
                \umlusecase[x=6, fill=green!20]{use case5}
                \umlusecase[x=6, y=-4]{use case6}
            \end{umlsystem}

            \umlactor{user}
            \umlactor[y=-3]{subuser}
            \umlactor[x=14, y=-1.5]{admin}

            \umlinherit{subuser}{user}
            \umlassoc{user}{usecase-1}
            \umlassoc{subuser}{usecase-2}
            \umlassoc{subuser}{usecase-3}
            \umlassoc{admin}{usecase-5}
            \umlassoc{admin}{usecase-6}
            \umlinherit{usecase-2}{usecase-1}
            \umlVHextend{usecase-5}{usecase-4}
            \umlinclude[name=incl]{usecase-3}{usecase-4}

            \umlnote[x=7, y=-7]{incl-1}{note on include dependency}
        \end{tikzpicture}
    \end{adjustbox}
    \caption{Use Cases Diagram}\label{fig:use-cases}
\end{figure}

\section{Class Diagram}
\lipsum[4-5]

\begin{figure}[H]
    \centering
    \begin{adjustbox}{width=0.8\textwidth, keepaspectratio}
        \begin{tikzpicture}
            \begin{umlpackage}{p}
            \begin{umlpackage}{sp1}
            \umlclass[template=T]{A}{
            n : uint \\ t : float
            }{}
            \umlclass[y=-3]{B}{
            d : double
            }{
            \umlvirt{setB(b : B) : void} \\ getB() : B}
            \end{umlpackage}
            \begin{umlpackage}[x=10,y=-6]{sp2}
            \umlinterface{C}{
            n : uint \\ s : string
            }{}
            \end{umlpackage}
            \umlclass[x=2,y=-10]{D}{
            n : uint
            }{}
            \end{umlpackage}

            \umlassoc[geometry=-|-, arg1=tata, mult1=*, pos1=0.3, arg2=toto, mult2=1, pos2=2.9, align2=left]{C}{B}
            \umlunicompo[geometry=-|, arg=titi, mult=*, pos=1.7, stereo=vector]{D}{C}
            \umlimport[geometry=|-, anchors=90 and 50, name=import]{sp2}{sp1}
            \umlassoc[arg=tutu, mult=1, pos=0.8, recursive=30|60|2cm]{D}{D}
            \umlinherit[geometry=-|]{D}{B}
            \umlnote[x=2.5,y=-6, width=3cm]{B}{Je suis une note qui concerne la classe B}
            \umlnote[x=7.5,y=-2]{import-2}{Je suis une note qui concerne la relation d'import}
        \end{tikzpicture}
    \end{adjustbox}
    \caption{Class Diagram}\label{fig:class-diagram}
\end{figure}

\section{Sequence Diagram}
\lipsum[6]

\begin{figure}[H]
    \centering
    \begin{adjustbox}{width=0.8\textwidth, keepaspectratio}
        \begin{tikzpicture}
            \begin{umlseqdiag}
                \umlactor[class=A]{a}
                \umldatabase[class=B, fill=blue!20]{b}
                \umlmulti[class=C]{c}
                \umlobject[class=D]{d}
                \begin{umlcall}[op=opa(), type=synchron, return=0]{a}{b}
                \begin{umlfragment}
                \begin{umlcall}[op=opb(), type=synchron, return=1]{b}{c}
                \begin{umlfragment}[type=alt, label=condition, inner xsep=8, fill=green!10]
                \begin{umlcall}[op=opc(), type=asynchron, fill=red!10]{c}{d}
                \end{umlcall}
                \begin{umlcall}[type=return]{c}{b}
                \end{umlcall}
                \umlfpart[default]
                \begin{umlcall}[op=opd(), type=synchron, return=3]{c}{d}
                \end{umlcall}
                \end{umlfragment}
                \end{umlcall}
                \end{umlfragment}
                \begin{umlfragment}
                \begin{umlcallself}[op=ope(), type=synchron, return=4]{b}
                \begin{umlfragment}[type=assert]
                \begin{umlcall}[op=opf(), type=synchron, return=5]{b}{c}
                \end{umlcall}
                \end{umlfragment}
                \end{umlcallself}
                \end{umlfragment}
                \end{umlcall}
                \umlcreatecall[class=E, x=8]{a}{e}
                \begin{umlfragment}
                \begin{umlcall}[op=opg(), name=test, type=synchron, return=6, dt=7, fill=red!10]{a}{e}
                \umlcreatecall[class=F, stereo=boundary, x=12]{e}{f}
                \end{umlcall}
                \begin{umlcall}[op=oph(), type=synchron, return=7]{a}{e}
                \end{umlcall}
                \end{umlfragment}
            \end{umlseqdiag}
        \end{tikzpicture}
    \end{adjustbox}
    \caption{Sequence Diagram}\label{fig:sequence-diagram}
\end{figure}

\section{Conclusion}
\lipsum[7][1-15]